% -*- coding: utf-8 -*-

\chapter{最近のデータベースシステム関係の話}
% * (アスタリスク)付きの \chapter* コマンドは原則不可とする

\begin{flushright}
 はやみず
\end{flushright}

この記事では、最近のデータベースシステムに関連する新しい話題を取り上げな
がら、その周辺の技術に関して思うことをだらだらと書き連ねてゆきます。あま
りまとまりがありませんが悪しからず。

\section{NoSQL}

流行りも一段落した感がありますが、まずはNoSQLの話から始めましょう。

NoSQLという言葉の定義自体が曖昧ではあるのですが、Not Only SQLを略して
NoSQL、つまり従前のSQLを使ったデータベースシステムではなく、それ以外の色々
なデータベース関連ソフトウェアのことを指します。

データベース分野においては、歴史的には``One size fits all''というスローガ
ンで語られることが多いのですが、RDBMSで全部やりましょうという時代が続いて
きました。ところが、データの量が増え、Webやモバイル端末の普及に伴ってユー
ザが増え、RDBMSでは対応しきれない場面が徐々に増えてきます。データベースの
大御所 Michael Stonebraker が2005年に ``One size fits all'' はもはや現実
的に無理が出てきているので、特定用途ごとに専用のデータベースエンジンを作っ
ていく必要がある、ということを指摘します。その動きとしてカラム志向型デー
タベースシステム、主記憶データベースシステムなどがあり、そのうちの一つに
NoSQLを位置づけることができるかと思います。



NoSQLという呼び方が非常に紛らわしいのですが、NoSQLと呼ばれている MongoDB、
CouchDB、Cassandra、Redis、memcached等々のソフトウェアと、従前のSQLベース
のデータベースシステムは直接対比して語るべきではない、というところに注意
が必要です。それどころか、NoSQLとひとくくりにされている諸々のソフトウェア
たちも、それぞれが同列に語られるべきものではありません。「これからの自体
SQLではなくNoSQLだ!」とか言っていると、「食べログなんてもう時代遅れ、時
代は塩スイーツ!」みたいに意味のわからない人になってしまうので注意しましょ
う\footnote{一方で、SQLとNoSQLが同列に語られることが多いのは、おそらく両
者を混同してほしいと思っている人たちがいるのではないかと思います}。

\section{Hadoop}

Hadoop、流行ってますね。

2004年にGoogleがMapReduceを発表し、

\section{はじめに}

\lettrine{ハ}
チロク世代がTwitter上で大々的に追悼されたのは記憶に新しい\footnote{「ハチロク世代追悼」で検索されたし}。かつてはIT業界を変えると期待されていた若者たちは、あるものは就職し、あるものは結婚し、そしてまたあるものは子供を持ち、別段IT業界を変えるでもなく普通のおっさんになろうとしている。もはやハチロク世代はアラサーであり、女子高生からはおっさんと呼ばれる年齢になってしまったのである。


