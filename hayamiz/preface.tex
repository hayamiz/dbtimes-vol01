% -*- coding: utf-8 -*-

\chapter*{まえがき}
\addcontentsline{toc}{chapter}{まえがき}
\thispagestyle{plainhead}

The Database Times vol.1 をお手にとっていただきありがとうございます。

本書は、データベースシステムを中心として、様々な情報技術に関する話題を読
者の皆様にお届けすることを目的としています。みなさんは「データベースシス
テム」というものにどのようなイメージをお持ちでしょうか。SQLを投げるとデー
タを返してくれる、なんだかよくわからないけれど枯れた時代遅れの技術だとい
うイメージを持っている人が多いのではないかと思います。

しかしながら、実際のデータベースシステムは古の技術を土台として、その時代
の最先端の技術が詰め込まれた非常に魅力的な技術の集大成なのです。世界中で
生み出されるデータ量がどのくらいかご存知でしょうか。2010年までに生み出さ
れたデータ量はなんと1,227ペタバイト。しかも、2020年には7,910,000ペタバイ
トになるであろうと予測されています。人類が産み出そうとしているデータ量は、
まさに爆発的な勢いで増え続けています。このデータの膨張を支える縁の下の力
持ちがデータベースシステムです。加速するデータ量の増加に伴って、データベー
スも日々進歩を続けているのです。

そんなデータベースシステムが、どのようにして生まれ、そして最近ではどんな
ことが話題になっているのか、データベースシステムの今と昔に迫るのが本書の
目指すところです。

\begin{flushright}
 2012年 8月

Hotchpotch Society
\end{flushright}

\newpage

\subsection*{お品書き}

\noindent {\bf ■ データベースシステムの夜明け}

今の世の中、データベースシステムといえばリレーショナルデータベースシステ
ムです。そのリレーショナルデータベースシステムが、一体どのようにして生み
出されたのか、そしてどのように発展を遂げていったのか、その歴史を辿ります。

\vspace*{\Cvs}

\noindent {\bf ■ PostgreSQLカンファレンス2012 レポート}

オープンソースDBMS代表格の一つであるPostgreSQLは、最近は業務でも用いられ
ることがかなり増えてきているようです。そんな PostgreSQL の最新技術事情に
触れられる PostgreSQL カンファレンスの参加レポートをお届けします。

\vspace*{\Cvs}

\noindent {\bf ■ とある世界で一番高速なBrainfuckインタプリタ}

近年ではSAP HANAやMemSQLなどの主記憶データベースが話題となっているように、
膨大するデータ量を捌ききるための「速さ」が強く求められています。データベー
スシステムはSQLというプログラミング言語の処理系という側面も持ちあわせてお
り、言語処理系の高速な実装はデータベースシステムには必要不可欠です。言語
処理系をいかに高速にするのか、その技術について世界最速Brainfuckインタプリ
タの実装者が解説します。

\vspace*{\Cvs}

\noindent {\bf ■ クラウド時代のDNS}

\noindent {\bf ■ IPv4がこの先生きのこるには}

最近では「クラウド」という言葉が話題となっていますが、その背景にはもはや
単一のコンピュータだけでシステムを構築してすべてがうまくいく時代は終焉を
迎え、ネットワークによりコンピュータを有機的に結合することが必須となりつ
つある現代の技術トレンドがよみとれます。ネットワーク技術では今一体何が起
きているのか、その最前線に迫ります。

\vspace*{\Cvs}

\noindent {\bf ■ 電算機技能者的同人誌執筆環境構築概論}

本書は \LaTeX で執筆されています。プログラマが \LaTeX で本を書くというの
はどういうことなのか、その開発(執筆)環境を紹介します。


\thispagestyle{plainhead}
