% -*- coding: utf-8 -*-

\chapter*{まえがき(仮)}
\addcontentsline{toc}{chapter}{まえがき}

\lettrine{ま}\ えがき。前書き。前書きとは、書物や論文などで本文の前に書
き添える文のことである。その書物の品位、格式などにより序文、端書き、巻頭
言などの言葉が用いられる。英語では preface、foreword などがそれに相当す
る言葉である。

そもそも、まえがきとは一体なんなのか、改めて考えてみたい。多くの人が本を
手にとったとき、必ずまず目にするのが表紙であり、その次に高い確率で読まれ
るのがまえがきである。つまり書物を人間に例えるとするならば、表紙とは外見
であり、まえがきとは開口一番その人間が初対面の相手に向かって話す内容であ
る。人は見た目が九割というベストセラーがあったが、残りの一割のうち九分を
決定するのが、このまえがきであろう。

であるとするならば、一体まえがきには何を書くべきなのだろうか。

教科書のような真面目な書物であれば、ある学問領域におけるその書物の占める
位置づけや、読み進め方、学習の進め方などが適切な内容であろう。著者によっ
ては、その書物を執筆した際のエピソードが語られることもしばしばみられる。
また欧米の場合には、重厚な教科書に限って愛する妻と娘に捧げられていること
が多いあたり、日本との文化の違いが伺われる。話が逸れたが、真面目な書物のま
えがきは、やはり原則として真面目にその書物の位置づけを語り、そして関係各
所への謝辞で締め括るのが一般的であり、常識的である。

一方、本書は別段真面目でもない同人誌である。何か学問的な意義を求めて出版
するわけでもなく、社会の役に立つことを良しとする類の書物でもない。書いて
字のごとく、同人(同好の士)が集まり、自らが良しとする内容を、自らの満足
のために書き連ねるだけの代物である。即ち、同人誌という書物そのものが自己
満足のために存在しているということであり、まえがきもまたその自己満足の一
部にすぎないのである。

これを以て、この一見無意味に思われるまえがきが、実は同人誌の目的である自
己満足という観点からすると、著者が満足しているという点において意味のある
ものである、という言い訳として、このまえがきを締めたいと思う。

\begin{flushright}
 2012年 8月

はやみず
\end{flushright}
