% -*- coding: utf-8 -*-

\cleardoublepage
\plainifnotempty

\chapter{データベースシステムの夜明け}

\begin{flushright}
はやみず
\end{flushright}

\lettrine{デ}ータベースシステム無しでは、今日の社会は成り立たないと言って
良いでしょう。社会が高度に情報化された現在、世界中で大量のデジタルデータ
が日々生み出され、飛び交い、消費され、そして蓄積されてゆきます。2011年に
人類の生み出したデータ量は1,800ペタバイトにのぼります。一方で、膨大なデジ
タルデータは、単に生み出され、蓄積されてゆくだけではただのゴミも同様です。
必要なときに必要なデータが取り出せるよう、適切に管理してゆかなければなり
ません。そのための根幹たる存在が、データベースシステムなのです。

みなさんが馴染みのあるであろう MySQL や PostgreSQL、あるいは Oracle といっ
たデータベースシステムは、正確には「リレーショナルデータベースシステム」
と呼ばれています。リレーショナルデータベースシステムの歴史を辿ると、その
起源はある一篇の論文に遡ることができます。

その論文こそが、Edgar F.  Coddにより1970年に発表された ``A Relational
Model of Data for Large Shared Data Banks'' です。この論文は、リレーショ
ナルデータベースの最も重要な基礎となる{\bf リレーショナルデータモデル}を
提唱したもので、いわばデータベースシステム分野における金字塔です。古典力
学を Newton が拓き、相対性理論を Einstein が拓いたとするならば、データベー
スにおける一大分野であるリレーショナルデータベースを拓いたのは間違いなく
Edgar Codd その人といって間違いないでしょう。

そして、Codd によりリレーショナルモデル提唱から数年の後に、UNIXで動作する
世界初のリレーショナルデータベースシステムの開発プロジェクトが立ち上がり
ます。Michael Stonebraker率いる{\bf INGRESプロジェクト}です。Coddにより確
立されたリレーショナルデータベースシステムの基礎理論を、実際に動くソフト
ウェアとして実現し、そしてそれを世に広めたのがINGRESなのです。

本稿では、Coddによるリレーショナルデータモデルの提唱から、INGRESプロジェ
クトの黎明期の記録を辿り、現代の社会を支えるデータベースシステムがいかに
して創り上げられたのか、その歴史を紐解いてみようと思います。

{\small ※ これ以降、特に断りのない場合、リレーショナルデータベースシステムを指して
単にデータベースシステムと書くことがあります。}

\section{時代はリレーショナルへ}

さようならCODASYL

\section{データベースシステムの産声}

UC Berlekey で INGRESプロジェクト始まる

System R ?

\section{つぎに}

\lettrine{あ}
いうえおかきくけこ。あいうえおかきくけこ。あいうえおかきくけこ。あいうえおかきくけこ。あいうえおかきくけこ。あいうえおかきくけこ。あいうえおかきくけこ。あいうえおかきくけこ。あいうえおかきくけこ。あいうえおかきくけこ。あいうえおかきくけこ。あいうえおかきくけこ。あいうえおかきくけこ。あいうえおかきくけこ。あいうえおかきくけこ。あいうえおかきくけこ。あいうえおかきくけこ。あいうえおかきくけこ。あいうえおかきくけこ。あいうえおかきくけこ。あいうえおかきくけこ。あいうえおかきくけこ。あいうえおかきくけこ。あいうえおかきくけこ。あいうえおかきくけこ。あいうえおかきくけこ。あいうえおかきくけこ。あいうえおかきくけこ。あいうえおかきくけこ。あいうえおかきくけこ。あいうえおかきくけこ。あいうえおかきくけこ。あいうえおかきくけこ。あいうえおかきくけこ。あいうえおかきくけこ。あいうえおかきくけこ。あいうえおかきくけこ。あいうえおかきくけこ。あいうえおかきくけこ。あいうえおかきくけこ。あいうえおかきくけこ。あいうえおかきくけこ。あいうえおかきくけこ。あいうえおかきくけこ。あいうえおかきくけこ。あいうえおかきくけこ。あいうえおかきくけこ。あいうえおかきくけこ。あいうえおかきくけこ。あいうえおかきくけこ。あいうえおかきくけこ。あいうえおかきくけこ。あいうえおかきくけこ。あいうえおかきくけこ。あいうえおかきくけこ。あいうえおかきくけこ。あいうえおかきくけこ。あいうえおかきくけこ。あいうえおかきくけこ。あいうえおかきくけこ。あいうえおかきくけこ。あいうえおかきくけこ。あいうえおかきくけこ。あいうえおかきくけこ。あいうえおかきくけこ。あいうえおかきくけこ。あいうえおかきくけこ。