% -*- coding: utf-8 -*-

\cleardoublepage
\plainifnotempty

\chapter{データベースシステムの夜明け}

\begin{flushright}
はやみず
\end{flushright}

\lettrine{デ}ータベースシステム無しでは、今日の社会は成り立たないと言って
良いでしょう。社会が高度に情報化された現在、世界中で大量のデジタルデータ
が日々生み出され、飛び交い、消費され、そして蓄積されてゆきます。2011年に
人類の生み出したデータ量は1,800ペタバイトにのぼります。一方で、膨大なデジ
タルデータは、単に生み出され、蓄積されてゆくだけではただのゴミも同様です。
必要なときに必要なデータが取り出せるよう、適切に管理してゆかなければなり
ません。そのための根幹たる存在が、データベースシステムなのです。

みなさんが馴染みのあるであろう MySQL や PostgreSQL、あるいは Oracle といっ
たデータベースシステムは、正確には「リレーショナルデータベースシステム」
と呼びます。リレーショナルデータベースシステムの歴史を辿ると、その起源は
ある一篇の論文に遡ることができます。

その論文こそが、エドガー・コッドにより1970年に発表された ``A Relational
Model of Data for Large Shared Data Banks'' です。この論文は、リレーショ
ナルデータベースの基礎となる{\bf リレーショナルデータモデル}を提唱したも
ので、いわばデータベースシステム分野における金字塔と言ってよいでしょう。
古典力学をニュートンが拓き、相対性理論をアインシュタインが拓いたとするな
らば、データベースにおける一大分野であるリレーショナルデータベースを拓い
たのは間違いなくエドガー・コッドその人です。

そして、コッドによりリレーショナルモデル提唱から遅れること数年、リレーショナルデータモデル

\section{データベースシステムは誰が作ったのか}

\lettrine{世}の中で広く用いられているデータベースシステムは、一体誰が創り
上げたのか、皆さんご存知であろうか。もちろんデータベースシステムというの
は非常に複雑なソフトウェアの集合体であり、非常に多くの人々の努力が結実し
たプロダクトである。その中で、今日のデータベースシステムが出来上がるため
のコアとなる部分を作ったのは誰か、という話である。

リレーショナルデータモデルの父、エドガー・コッド。そしてINGRESの父、マイ
ケル・ストーンブレーカー。

リレーショナルデータモデルにより、データベースシステムを理論的に扱うこと
を可能とし、データベースシステムをサイエンスにならしめたのがエドガー・コッ
ドであり、そしてリレーショナルデータモデルをいち早く現実のシステムに適用
し、リレーショナルデータベースシステムの歴史を切り開いたのがマイケル・ス
トーンブレーカーである。

\subsection{リレーショナルモデルとエドガー・コッド}



\section{つぎに}

\lettrine{あ}
いうえおかきくけこ。あいうえおかきくけこ。あいうえおかきくけこ。あいうえおかきくけこ。あいうえおかきくけこ。あいうえおかきくけこ。あいうえおかきくけこ。あいうえおかきくけこ。あいうえおかきくけこ。あいうえおかきくけこ。あいうえおかきくけこ。あいうえおかきくけこ。あいうえおかきくけこ。あいうえおかきくけこ。あいうえおかきくけこ。あいうえおかきくけこ。あいうえおかきくけこ。あいうえおかきくけこ。あいうえおかきくけこ。あいうえおかきくけこ。あいうえおかきくけこ。あいうえおかきくけこ。あいうえおかきくけこ。あいうえおかきくけこ。あいうえおかきくけこ。あいうえおかきくけこ。あいうえおかきくけこ。あいうえおかきくけこ。あいうえおかきくけこ。あいうえおかきくけこ。あいうえおかきくけこ。あいうえおかきくけこ。あいうえおかきくけこ。あいうえおかきくけこ。あいうえおかきくけこ。あいうえおかきくけこ。あいうえおかきくけこ。あいうえおかきくけこ。あいうえおかきくけこ。あいうえおかきくけこ。あいうえおかきくけこ。あいうえおかきくけこ。あいうえおかきくけこ。あいうえおかきくけこ。あいうえおかきくけこ。あいうえおかきくけこ。あいうえおかきくけこ。あいうえおかきくけこ。あいうえおかきくけこ。あいうえおかきくけこ。あいうえおかきくけこ。あいうえおかきくけこ。あいうえおかきくけこ。あいうえおかきくけこ。あいうえおかきくけこ。あいうえおかきくけこ。あいうえおかきくけこ。あいうえおかきくけこ。あいうえおかきくけこ。あいうえおかきくけこ。あいうえおかきくけこ。あいうえおかきくけこ。あいうえおかきくけこ。あいうえおかきくけこ。あいうえおかきくけこ。あいうえおかきくけこ。あいうえおかきくけこ。