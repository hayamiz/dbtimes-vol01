% -*- coding: utf-8 -*-
%!TEX root = ../book.tex

\cleardoublepage
\plainifnotempty

\chapter{とある世界で一番高速なBrainf*ck処理系}
\begin{flushright}
hogelog
\end{flushright}

\section{まえがき}
\lettrine{世}
の中にはたくさんのプログラミング言語が存在する。
そしてたくさんのプログラミング言語処理系が存在する。
プログラミング言語、そして言語処理系にとって重要なものは非常に多い。
目的のプログラムを書けるだけの言語の記述力、
効率の良いメモリ管理システム、
堅牢な例外処理機構など、挙げればキリがない。
何を重視するかは議論の別れるところであろうが、
{\LARGE \textbf{高速な言語処理系の存在}}
が重要であることは間違いがないだろう。

Brainf*ck
\footnote{http://www.kmonos.net/alang/etc/brainfuck.php とかわかりやすい紹介です}
というとてもすごく超オモシロかっこいい言語があるのですが
どの処理系を使えばいいのかわからなかったのでとりあえず自分で
世界最速の処理系を作ってやったぜ、と書こうかと思いましたがやめました。
ここではBrainf*ckの「処理系の実装が容易である」という特徴に着目します。
というわけでBrainf*ckの処理系をお題として言語処理系の高速化に
取り組んでみましょう。
とりあえずの目標はこの記事のタイトル通りに
{\LARGE \textbf{「世界で一番高速なBrainf*ck処理系」}}
ということにしましょう。

なんかこんなかんじ。

\section{高速化手法}
まあなんか色々する。

\subsection{direct threading}
ruby maniacs紹介程度で済ませていいんじゃないかな。

\subsection{JITコンパイル}
Xbyakとか。

\subsection{最適化}
いろんなパターンを紹介。

締め切りちけーーーー!!!!!!