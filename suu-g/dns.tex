% -*- coding: utf-8 -*-
%!TEX root = ../book.tex

\cleardoublepage
\plainifnotempty

\chapter{クラウド時代のDNS}
\begin{flushright}
suu-g
\end{flushright}

%\section{Welcome to DNS world}
\lettrine{ヰ}
ンターネットが始まってから20年、ARPANETから数えると40年以上。
インターネットはここ十年ほどで一気に生活に不可欠のものとなりました。
そんなインターネットを支える基礎技術の一つがDNSです。

そんなDNSを取り巻く状況ですが、今年は随分と熱い一年間でした。
DNSの最初のRFC 1034, 1035 が出たのは1987年でしたが、それから25年
経過した今でもなお新たな問題が発覚し続ける、枯れているようで
枯れていない少し枯れた技術DNS。その現状と問題を、プロトコルを
確認しながら追っていくのがこの記事です。

とはいえ、普通のソフトウェア屋さんにとってDNSとは使うものであって
それ以外ではないのでしょうから、こちらの世界のニュースについては
あまり確認されていないことと思います。

今年は、どんなことが起きたのでしょうか?
DNS界隈で起きた問題について羅列してみます。
\begin{itemize}
  \item (日本)児童ポルノに関するDNSブロッキング立法・ISPにおける運用が開始
  \item SOPAにGo Daddyが賛成を表明するも一日で撤回
  \item SOPAに反対を表明するためWikipediaが24時間閉鎖する
  \item SOPAに反対を表明してAnonymousがOperation Blackoutを表明
  \item フレッツ網におけるIPv6-IPv4フォールバック問題の解決法として、国内各所でAAAAフィルタリングが検討される
  \item Google, AAAAフィルタを導入している可能性のあるDNSサーバに対してAAAAレコードを回答しない運用を宣言
  \item 浸透言うな!
  \item Ghost Domain問題の発表
  \item ニフティクラウドのDNS障害問題
  \item GMOクラウドのDNSで謎の障害がある
  \item さくらのDNSに(サブ)ドメインジャッキングが可能な脆弱性が発覚
  \item JPRSがあらためてコンテンツサーバとキャッシュサーバの兼用の危険性を告知
  \item BIND, 長さ0のRDATAにより異常終了する脆弱性(重複をお許しください)
  \item お名前.com、ninja toolsのNSレコードを規約に基づき変更
  \item 今年の夏のDNS祭りはBINDではなくNSD
\end{itemize}

豊作ですね。
問題に限定しなければ他にも、DNSSEC.jpが活動を終了したり、
gTLDの申請があったりと、今年は本当に色々なことがありました。

今回は、「クラウド」時代となった今、DNS界隈でどのような
問題が起きているかを振り返ります。

\section{DNSとぼくらの世界}

DNSのパケットフォーマットは、RFC1034,1035 に載っている通りです。
DNSパケットには四つのセクションがあり、それぞれQuery, Answer, Authority,
それからAdditionalと呼ばれています。
\footnote{http://www.ietf.org/rfc/rfc1034.txt Section 3-7}
\footnote{http://www.ietf.org/rfc/rfc1034.txt Section 4-1}
それぞれのセクションに入ったデータの意味を適切に解釈することで、DNSの
メッセージ伝達は動作しています。

これをJSON風に書くと、図1のようになってます。(正確には、DNSSECやEDNS0対応
等でフラグはもう幾つかある)
読者の皆様におかれましてはこのパケットの意味を判断しようとしてくれると
期待しますが、勘の良い人だと、このプロトコルのもつ怪しさをなんとなく
感じ取ったかも知れません。

さて、このようなパケットを元にして、DNSというシステムは成り立っています。
このパケットを用いて Paul Vixie らが作ろうとしたのが、巨大な名前空間の木です。
そして、その木をデータベースとしてユーザから参照できるようにした。それが
DNSというシステムです。あ、ご存知ですよね。すいません。でも大事なこと
なんで。

ここでnoteしておきたいのが、DNSプロトコルが作られた当時の時代情勢です。
DNSは、/etc/hostsからの移行のひとつの可能性として生まれていたわけですが、
その時代が今とは違っていて、例えばDNSでちょっと間違いが起こったとしても、
それは手動で直していれば良かった。今じゃ警察沙汰だもん。もぉーちょっと
お名前.comさん。NS勝手に向けないでー。

もっとも、DNSに関しては今でも法的にきちんとしているとは言い難いところです。
DNSでのブロッキングの効果を法律で認めているわりには、JPRSが電気通信事業者
というわけでもありませんし。このあたりは、そろそろ本格的に議論されるべき
話ですよね。

さて話は戻ります。いま当時の時代情勢を取り上げたのは、DNSという
分散システムが、設計段階から「異なる管理者同士で」ひとつの巨大な名前空間を
管理するように作られている、ということを再認識したかったからです。
各管理者が責任を持つ名前空間の範囲をゾーンと呼び、その空間の管理を他の人に
任せることを委譲と呼んでいる。そういう信頼関係となっています。
それらを前提として、内容に入って行きましょう。


\section {DNSの正常系}

DNSの正常系は、先に示しました4つのセクションと5つのフラグを使い、
おおよそ次の5種類のパケットを用います。
\begin{enumerate}
  \item Query (recursive)
  \item Query (not recursive)
  \item Response (Authoritative, with glue record in additional section)
  \item Response (with Answer)
  \item Response (against recursive query)
\end{enumerate}
非常にフリーダムです。

ユーザがキャッシュサーバに対して問い合わせを行い、そのキャッシュサーバ
がユーザの代わりにイテラティブにクエリを発行するときのパケットは、
図のようになります。

キャッシュサーバがキャッシュを持ったあとは、1と5のパケットのみの
やりとりとなります。ここまでは皆様ご存じですねー。

また、このキャッシュサーバがコンテンツサーバも兼ねている場合、
図のようなレスポンスが返ってくることになります。
ユーザからは、これはAAビットが立って(権威のある回答として)見えます。
なお、権威のある回答は通常のキャッシュと比較して優先されるべき
ものということになっていますが、ユーザには大差ありません。
\footnote{http://www.ietf.org/rfc/rfc2181.txt}

正常系としては、こんなところでしょう。これだけの機能があれば、
ルートサーバを頂点とする綺麗なツリー構造が世界に作れます。
すばらしい!DNS最高!いあ!いあ!


\section { 異常系 }
以上のようなDNSですが、問題が山積みです。これはどう考えても
プロトコルが悪いのです。

\subsection{ Lame Delegation }
これは基本だよなあ。基本が火を噴くか否か、そこのスリルを味わってみたいなあ。
要は、上位のサーバでの正しくない設定です。
責任者の違いはありますが、構造的には、 Visaドメイン問題 
\footnote{http://www.e-ontap.com/summary/}
と言われているものもこれのうちのひとつと数えてしまって良いでしょう。


\subsection{ 浸透さんって幽霊じゃね? }

DNSの設定変更や引っ越しをしたときに、その結果が「浸透」
するのを待たなければいけない…というのは有名な嘘ですが、
その「浸透」現象自体は実際に起こっていて、それもサーバ
の実装バグのせいだった、と言うことが最近分かりました。
\footnote{www.isc.org/files/imce/ghostdomain\_camera.pdf}
これが「幽霊ドメイン」です。 
具体的な問題内容については、DNSOpsの資料 
\footnote{http://dnsops.jp/bof/20120425/20120425-DNSOPS.jp-BoF\_Ghost\_Domain\_Names\_v02.pdf}
や「Geekなぺーじ」
\footnote{http://www.geekpage.jp/blog/?id=2012/3/21/1} に詳しい
のでそちらをご参照ください。

ソフトウェア的に考えたときに問題となるのは、古いキャッシュ
が残存して昔のサーバへ問い合わせを続けるということ、
それから、問い合わせを続けていると古いキャッシュがexpire
しないという実装バグがあったということ。そして最後に、
本来のDNSのツリーから外れたコンテンツサーバが、レコードを
消さずに残しているために、昔のレコードをいつまでも返し続ける
場合があるということ。この三点です。

本来の信頼関係から外れたところのデータが有効性を持ってしまい、
おまけにその状態を意図的に作り出せる、という状態ですね。
このような問題が今年までバグとして存在し続けていたわけです。

システムとして考えると、キャッシュをTTL分保持することは
至極もっともなようですが、信頼関係のツリーまで含めてキャッシュすると
いうことは、非常に責任が大きい行為です。
今回のように、「なぜかキャッシュが消えない」という
異常事態になったとき、影響が大きくなります。


\subsection{ サブドメインジャッキング問題 }
2011年6月、さくらのDNSサービスにおいて、一時的にサブドメイン
ジャッキングが可能になっていた問題が発覚しました。
これは、 example.jp ドメインが登録されていたとしても、別のユーザに
よって自由に www.example.jp ドメインを作成できてしまった、という
問題です。
\footnote{さくらはたまたま注目を浴びただけで、問題を抱えたサービス
プロバイダは他にもたくさんある…が、その話はまた今度}

これは、DNSの本質的な問題を幾つか激しく突いています。

ひとつには、DNSの作成しているツリーが理想的なものではないこと。
あるゾーンを委譲する先のサーバは、別ホストになっているとは限らず、
同じホストである可能性があります。
そのため、フルリゾルバが正しく委譲されている先のサーバにたどり着く
前に、どこかの「ゾーンを持っていると勝手に宣言している」だけの
サーバによって偽の情報を掴まされる可能性があります。

ふたつめとして、ゾーンと言う概念が、プロトコルや実装とかけ離れて
しまっているという点があります。
ドメインのデリミタはドットですが、ドットがあったからと 言って
ゾーンが違っているかどうかは不明です。
example.jp ゾーンを他へ委譲したサーバ上で www.example.jp のAレコードを
登録できなくなるわけではありませんし、www.example.jpを問い合わせた
ときにどちらが返って来るかは実装依存(!) になってしまっています。


最後に、フルリゾルバがどのコンテンツサーバにも同じクエリを投げる、
というプロトコルの怪しさがあります。
これは、どこでゾーンが分かれるかが問い合わせを行うまでは
キャッシュサーバには知ることができない、ということとは符合します。
しかし、キャッシュサーバはゾーンの別れ方について【知らないでは済まされない】。
問い合わせを行ったキャッシュサーバは即座にその結果をキャッシュする
ことを実際に行っています(RFC的には、「キャッシュしても良い」)。
キャッシュサーバは、一種類の問い合わせを行ったときの結果によって、
様々な動作を行います。これは、最終的にコンテンツを持っているサーバから問い合わせる
ために用いるパケットも、そのサーバを見つけるためのパケットも同じということで、
それらの違いはフルリゾルバが判断する必要があります。
どうせキャッシュサーバは状態を持つのだから、結果を見て処理をディスパッチ
するよりも、メッセージやパケットフォーマットや通信の口を変えた方が、
圧倒的に楽でバグが少なくなったでしょうね。

もしも問い合わせがゾーンごとに行われていれば、子ゾーンについて問い合わせて
いるところで孫ゾーン情報が返ってきたり、あるいはAレコード問い合わせ
に対して謎のNSレコードをキャッシュさせられたりすることは無かったでしょう。
どう見てもプロトコル上の脆弱性です。本当にありがとうございました。

これらの問題点を合わせて起こった問題が、このサブドメインジャッキング問題です。


\subsection{ キャッシュ兼用サーバ }
コンテンツ・キャッシュ兼用サーバはしばしば問題と言われます。
先に示したように、自由に偽ドメインを入れられるコンテンツサーバを
フルリゾルバとして使用するのは、 自分のホストの /etc/hosts の書き換え権を
他人に委ねているようなものだ、というのはお分かりいただけるかと思います。

さて、ソフトウェア的に見た時に、キャッシュ兼用サーバに潜むDNSの問題は
というと…、これは優先順位やレイヤの問題があります。
AuthoritativeなサーバからのAnswerセクションは、他のどのレコードよりも
優先されるべきだとRFC2181にありますが、毒入れの可能性を考えたときに
もっとも危険性が高いのも、このAuthoritativeなAnswerセクションです。
このりくつはおかしくない。

「DNSサーバの相乗りという業務形態自体が、プロトコルとの相性が悪いのではないか」
と言った方がいましたが、 その意味が分かるのではないでしょうか。


\section{まとめ}
DNSは今やインターネットに不可欠な存在となっていますが、大きな問題があります。
その問題をEDNS0やDNSSECなどを用いて解決しようとしているのが現在のDNS界隈
ですが、そもそもDNSというプロトコル自体に設計レベルでの問題が多すぎます。
皆様方におかれましては、このDNSという「想定外に利用されてしまった」
プロトコルの問題を他山の石とし、これからの分散システムの開発にぜひ
活かしていただければ、記事を書いた当人としては満足でございます。

以上、タイトル詐欺の記事でした。おあとがよろしいようで。

